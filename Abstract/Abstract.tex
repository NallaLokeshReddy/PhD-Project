\abstract
%
\noindent KEYWORDS: \hspace*{0.5em} \parbox[t]{4in}{Distribution static compensator, dual-output converter, dynamic voltage restorer, phase-locked loop (PLL), sliding mode control, unified power quality conditioner.}
%
\vspace*{30pt}
\\
%
Distributed generation (DG) is a constantly evolving trend that connects renewable energy sources (RES) like solar and wind to the utility grid, in addition to the conventional energy sources like coal, oil, and gas. The widespread use of RESs and the power-switching electronics devices at the industrial and residential levels create power quality (PQ) issues. The custom power devices (CPD) were developed at the distribution system as a result of the severe regulations imposed on DG systems and the consumer demand for better PQ. This thesis is focused on improvements to the sliding mode control schemes of CPDs under various conditions and configurations, and an overview is given below.

Distribution static compensator (DSTATCOM), in general, improves current-related PQ issues, while dynamic voltage restorer (DVR), in general, improves voltage-related PQ concerns. The back-to-back (BTB) configuration of the unified power quality conditioner (UPQC) combines DSTATCOM and DVR. As a result, UPQC is a versatile CPD that can strengthen the system against both current and voltage-related PQ difficulties. However, the DVR converter is almost idle most of the time, since the occurrence of voltage related issues is less often. Thus, the converter utilization is poor with BTB configuration. To improve the converter utilization factor without compromising the functionalities of UPQC, the reduced switch count converter: dual-output converter (DOC) is considered in this thesis for UPQC system. The control algorithms of DOC based systems should include the control of converter neutral-point voltage in addition to current and voltage control. Further, accurate extraction of fundamental frequency positive sequence (FFPS) components and phase angle of grid voltages are required for an improved performance of UPQC system.     

Therefore, the extraction of FFPS components based on the operators: second order generalized integrator (SOGI), cascaded delayed signal cancellation (CDSC) and multiple delayed signal cancellation (MDSC) are reviewed. The extraction accuracy of three operators are evaluated, for various grid conditions, based on the quality of frequency output from synchronous reference frame (SRF) phase-locked loop (PLL). The performance comparison of these three operators based PLLs is presented using simulation results. The CDSC operator is found with superior performance characteristics and hence it is considered for the control of UPQC in this thesis.  

The control algorithm can be implemented in any one of the reference frames: $dq$, $\alpha \beta$ and natural reference frame ($abc$). Due to availability of generation of reference voltages/currents in $abc$ frame, implementing a suitable controller in the same reference frame would reduce the computational burden and the signal transformation errors. However, the three-phase three-leg or four-leg voltage source converters suffer from the coupling issue in $abc$ frame. The dynamics of each state variable depends on the control inputs of all four legs of the converter. This coupling leads to difficulty in assigning appropriate values to the control inputs using a conventional sliding surface in sliding mode control (SMC) scheme. To address this, a new sliding surface is proposed for both four-leg converter based DSTATCOM and DVR systems. Simulation and experimental studies conducted on both systems are presented. The presented studies verify the performance of the proposed control scheme for both DSTATCOM and DVR systems. Further, the proposed scheme is extended for four-leg dual-output converter based UPQC and its performance verification is presented using simulation results. 
