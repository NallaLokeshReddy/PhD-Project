\chapter{CONCLUSIONS AND FUTURE SCOPES}
\label{7.Chap:Conclusions}
In this chapter, concise descriptions of the important contributions made in this thesis, along with conclusions and possibilities for future work, are presented.

\section{SUMMARY}
The interconnection of renewable energy sources and the increased utilization of non-linear loads, powered by semiconductor switching devices, has led to a decline in power quality (PQ) in distribution networks. Various factors such as symmetrical and asymmetrical faults, lightning, and environmental conditions further decrease the reliability of distributed generation (DG) systems, causing voltage variations such as sag, swell, and interruptions. Moreover, the non-linear loads draw distorted currents from the grid through the feeder impedance, leading to voltage distortions at the point of common coupling (PCC). This distortion can be harmful to sensitive loads connected at that point, potentially causing damage. To address these challenges, custom power devices (CPD), specifically the unified power quality conditioner (UPQC), have been introduced.

The conventional back-to-back (BTB) topology of the UPQC has been effective in addressing both current and voltage-related PQ issues. However, a drawback of the BTB configuration is that the series converter is idle for a significant portion of the time, as voltage-related problems occur less frequently. This leads to poor utilization of the converter. To enhance the converter utilization factor without compromising the UPQC's ability to generate the required voltages and currents, recent research has focused on developing a new configuration called the dual-output converter (DOC) based UPQC. In this topology, the traditional back-to-back arrangement is replaced by a reduced switch count converter topology, such as dual-output converter.

The control of custom power devices such as the UPQC requires a suitable controller capable of handling both fundamental and harmonic signals. Linear proportional-integral (PI) controllers are commonly used for tracking DC signals accurately. However, they are not sufficient for UPQC control as UPQC deals with fundamental and multiple harmonic signals. While proportional-resonant (PR) controllers can accurately track specific frequency signals for which the PR gains are tuned, they are limited to a single frequency. To process the wide range of frequencies encountered in UPQC applications, multiple PR controllers are often paralleled with a PI controller. This parallel arrangement increases design complexity. Alternatively, non-linear controllers are capable of handling signals at any frequency, provided the processor's sampling rate is sufficiently low. Therefore, non-linear controllers can be developed in any reference frame. Additionally, since UPQC contains non-linear components like power semiconductor switches, non-linear controllers are well-suited for UPQC control. Among the available non-linear controllers, sliding mode controller (SMC) is often considered due to its robustness against parameter variations and model mismatches.

These considerations have motivated the development of a complete control algorithm in the natural frame itself for the DOC based UPQC system. The unique topological characteristics of the DOC necessitate not only the control of compensator currents and voltages but also the control of neutral-point voltages (NPV). To achieve NPV control within the sliding mode controller framework, a novel sliding surface has been proposed for both the four-leg distribution static compensator (DSTATCOM) and the four-leg dynamic voltage restorer (DVR). To assess the performance of the proposed control scheme, extensive simulation and experimental studies have been conducted under various grid conditions for both the four-leg DSTATCOM and the four-leg DVR. These studies validate the effectiveness of the control scheme in regulating the compensator currents, voltages, and NPV, demonstrating its robustness and ability to enhance power quality.

Furthermore, the proposed control scheme has been extended to the DOC based UPQC-L configuration, where the shunt compensator is positioned on the left side (grid side) of the UPQC. A novel reference generation approach has been proposed for the neutral-point voltages in the UPQC-L application. Through simulation studies, the performance of the proposed control scheme for the four-leg DOC based UPQC-L has been evaluated, verifying its effectiveness and suitability for maintaining power quality and mitigating grid disturbances.

The DOC-based UPQC-L configuration offers the advantage of compensating voltage sags with reduced switch counts compared to the back-to-back (BTB) configuration. However, it has a limitation in its voltage swell compensation capability when using the same DC-link voltage as the BTB configuration. To enhance the voltage swell compensation capability of the DOC-based UPQC-L system, the DC-link voltage needs to be increased.

On the other hand, the BTB configuration provides a full range of voltage swell compensation capability but at the cost of additional switches. This means that the BTB configuration can effectively handle a wide range of voltage swells without limitations. However, this advantage comes with the drawback of requiring a higher number of switches, which increases the complexity and cost of the system.

Therefore, the choice between the DOC-based UPQC-L configuration and the BTB configuration depends on the specific requirements of the application. If the voltage swell compensation range is a critical factor and cost is not a major concern, the BTB configuration may be preferred. On the other hand, if reducing switch counts and cost are important considerations, the DOC-based UPQC-L configuration can be a suitable option, even though it has limitations in voltage swell compensation.

The major outcomes of the thesis are listed below:
\begin{enumerate}
	\item The phase angle of the grid voltages is determined using the traditional synchronous reference frame phase locked loop (SRF-PLL). However, it accurately computes angles only for fundamental balanced voltages. When dealing with disturbed voltages, the incorporation of a low-bandwidth controller is necessary, but this tends to result in a sluggish response from PLL. To overcome this challenge, various pre-filters such as second order generalized integrator (SOGI), cascaded delayed signal cancellation (CDSC) and multiple delayed signal cancellation (MDSC) have been reviewed and compared. Through the comparative analysis, it has been observed that CDSC is the preferred choice due to its requirement of fewer delay operations, lower computational intensity and accurate performance in handling various voltage disturbances.    
	\item While developing a conventional sliding mode control scheme for four-leg VSI based systems in the natural reference frame, it has been analytically demonstrated that the conventional scheme encounters a challenges in generating switching pulses. This difficulty arises due to the coupling of dynamics in the sliding variable on each phase through all phase pole voltages. 
	\item In order to tackle the inherent coupling issue among the phases, a novel sliding surface has been proposed. This surface is defined as the sum of DSTATCOM current and current resulting from the neutral point voltage (NPV) for current compensation. Additionally, for voltage compensation, the sliding surface is the sum of DVR voltage and voltage due to NPV. The effectiveness of the proposed control scheme has been validated through simulation and experimental studies on a four-leg DSTATCOM and DVR. Since the UPQC is a combination of DSTATCOM and DVR, the proposed control scheme is also applicable to conventional back-to-back (BTB) converter based UPQC.
	\item The series converter in BTB configuration remains inactive for a significant duration since voltage-related problems are less frequent. This results in suboptimal utilization of the converter. In order to improve the converter utilization factor without compromising the UPQC’s ability to generate the required voltages and currents, an analysis is conducted using a dual output converter (DOC) based UPQC-L with the proposed control scheme through simulation studies. However, its worth noting that the DOC based UPQC-L exhibits a limitation in its ability to compensate for voltage swells when employing the same DC-link voltage as the BTB configuration. To enhance the voltage swell compensation capability, it becomes necessary to increase the DC-link voltage.  
\end{enumerate}

\section{SCOPE FOR FUTURE RESEARCH}
The application of dual-out converter to power system applications can be further explored and enhanced in the following ways: 
\begin{itemize}
\item \textbf{Fault-Tolerant Feature:} Incorporating fault-tolerant features into the dual-output converter (DOC) topology is crucial for ensuring reliable operation of the UPQC or any other application based on the DOC. This can involve the development of fault detection, isolation, and recovery mechanisms to maintain system performance and stability in the presence of faults. Research efforts can focus on improving the fault-handling capabilities of the DOC-based UPQC to enhance the reliability of the system. 
    
\item \textbf{Integration of PV and/or Battery Storage Units:} The configuration of a DOC-based UPQC-L system can be extended to include the integration of photovoltaic (PV) and/or battery storage units. This investigation involves studying the performance of the proposed control scheme in a multi-functional UPQC that combines load voltage regulation, load balancing, and renewable energy integration. The objective is to optimize the utilization of PV and battery storage systems in conjunction with the DOC-based UPQC-L to enhance power quality, mitigate voltage fluctuations, and maximize the utilization of renewable energy sources.

\end{itemize}